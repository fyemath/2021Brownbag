% Options for packages loaded elsewhere
\PassOptionsToPackage{unicode}{hyperref}
\PassOptionsToPackage{hyphens}{url}
%
\documentclass[
]{article}
\usepackage{amsmath,amssymb}
\usepackage{lmodern}
\usepackage{iftex}
\ifPDFTeX
  \usepackage[T1]{fontenc}
  \usepackage[utf8]{inputenc}
  \usepackage{textcomp} % provide euro and other symbols
\else % if luatex or xetex
  \usepackage{unicode-math}
  \defaultfontfeatures{Scale=MatchLowercase}
  \defaultfontfeatures[\rmfamily]{Ligatures=TeX,Scale=1}
\fi
% Use upquote if available, for straight quotes in verbatim environments
\IfFileExists{upquote.sty}{\usepackage{upquote}}{}
\IfFileExists{microtype.sty}{% use microtype if available
  \usepackage[]{microtype}
  \UseMicrotypeSet[protrusion]{basicmath} % disable protrusion for tt fonts
}{}
\makeatletter
\@ifundefined{KOMAClassName}{% if non-KOMA class
  \IfFileExists{parskip.sty}{%
    \usepackage{parskip}
  }{% else
    \setlength{\parindent}{0pt}
    \setlength{\parskip}{6pt plus 2pt minus 1pt}}
}{% if KOMA class
  \KOMAoptions{parskip=half}}
\makeatother
\usepackage{xcolor}
\IfFileExists{xurl.sty}{\usepackage{xurl}}{} % add URL line breaks if available
\IfFileExists{bookmark.sty}{\usepackage{bookmark}}{\usepackage{hyperref}}
\hypersetup{
  pdftitle={A Practical Guide to Markdown and Tools},
  pdfauthor={Fei Ye},
  hidelinks,
  pdfcreator={LaTeX via pandoc}}
\urlstyle{same} % disable monospaced font for URLs
\setlength{\emergencystretch}{3em} % prevent overfull lines
\providecommand{\tightlist}{%
  \setlength{\itemsep}{0pt}\setlength{\parskip}{0pt}}
\setcounter{secnumdepth}{-\maxdimen} % remove section numbering
\ifLuaTeX
  \usepackage{selnolig}  % disable illegal ligatures
\fi

\title{A Practical Guide to Markdown and Tools}
\author{Fei Ye}
\date{Wednesday, December 1, 2021}

\begin{document}
\maketitle

\hypertarget{a-practical-guide-to-markdown-and-tools}{%
\section{A Practical Guide to Markdown and
Tools}\label{a-practical-guide-to-markdown-and-tools}}

\href{https://www.markdownguide.org/getting-started/}{Markdown} is a
lightweight markup language that can be used to add formatting elements
to plaintext text documents It is the popular markup language that
people use it to create websites, documents, books, and presentations.
For example, \href{https://yfei.page}{my website}, and those
\href{https://yfei.page/teaching/\#books}{online books and slides} were
built using markdown or its extension R markdown.

In this presentation, I will introduce the syntax of Markdown, the easy
to use editor \href{https://typora.io/}{Typora}, the universal document
converter \href{http://pandoc.org/}{Pandoc}, the static site generator
\href{https://gohugo.io/getting-started/quick-start/}{Hugo}, and the
cloud computing and website hosting platform
\href{https://app.netlify.com/drop}{Netlify}. If time permits, I will
also introduce the version control system
\href{https://git-scm.com/}{git}, the repository hosting platform
\href{https://github.com/}{GitHub}, the app
\href{https://desktop.github.com/}{GitHub Desktop}, and
\href{https://rmarkdown.rstudio.com/}{R Markdown}.

By the end of this talk, participants will be able to

\begin{enumerate}
\def\labelenumi{\arabic{enumi}.}
\item
  write in Markdown,
\item
  use Typora and Pandoc to create a webpage, a word document, a pdf
  file, and a presentation (html or ppt) from a Markdown file, and
\item
  use Hugo to create a website (or blogsite) and publish it.
\end{enumerate}

\end{document}
